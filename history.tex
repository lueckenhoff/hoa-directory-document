\section{A Brief History}

In 1936 John Sloan purchased the Owen Allen family farm.

\begin{quote}
It was under the ownership of John Sloan that FOXBORO saw its brightest
hour. Sloan was a principal in Nashville’s leading department store,
Cain-Sloan. He built a mansion in the style of an English manor on the
crest of the hill that overlooked his domain.

John Sloan was tall and courtly. Every inch a patrician gentlemen. He
named his farm Maple Grove. It was, first of all, a working farm where he
raised horses and operated a dairy with registered Guernsey cattle. His
cows won blue ribbons at the State Fair, and his horses won races at the
annual Steeplechase. (Note: Iroquois Memorial Steeplechase was co-founded
by John Sloan).

John Sloan carved a place for himself in the Brentwood community. No
single person has had greater influence on the way in which Brentwood
has developed. He was one of several men and their families who moved to
Brentwood in the 1930s and 1940s. They bought up the old plantations and
homes that had fallen into neglect. Well heeled, they restored Brentwood
to its pre-Civil War splendor. Sloan was in the lead group of men. He
is quoted as saying that; ‘People were attracted to Brentwood by the
three H’s –Homes, Horses, and Hounds’.

It was, indeed, a Renaissance for Brentwood. The Hillsboro Hounds were
organized. It was foxhunting group that rode the hills and farms of
Brentwood every Saturday afternoon, in Old English style, in chase of
the red fox. Riders were resplendent in their Pink Coats and riding
habits. The hounds were kenneled at Green Pastures, home of Mason
Houghland. They would gather on the front lawn of Sloan’s Concord Road
home for a blessing of the hounds by a local Episcopal priest.

Sloan and a group of visionary Brentwood citizens saw the need for
organized and systemic planning in Brentwood. They organized the Brentwood
Planning Commission before the City of Brentwood was incorporated. They
conceived of a plan for the development of Brentwood, which is still in
place today.

John Sloan sold 300 acres of his Maple Grove Farm for development. It
became FOXBORO.
\end{quote}

Vance Little details its origins in his book ``When Cotton Was King on
Concord Road, A History of Brentwood''
